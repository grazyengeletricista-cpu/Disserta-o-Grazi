% Página de Dedicatória
\cleardoublepage
\thispagestyle{empty}

\begin{center}

\textbf{DEDICATÓRIA}
    
\end{center}



\vspace*{\fill}

\hspace*{7cm}
\begin{minipage}{\dimexpr\textwidth-8.2cm\relax}
Dedico este trabalho em memória do meu pai, José Fernandes Monteiro, meu herói, que sempre me incentivou a continuar os estudos,
 e dedico também em memória da minha avó, Júlia Maria de Almeida Oliveira, que sempre me incentivou a correr atrás dos meus sonhos. Tenho saudades de tudo que passei ao lado de vocês. Como eu queria agora poder abraçá-los e agradecer por tudo, pois, sem vocês, este trabalho e muitos dos meus sonhos não se realizariam. Te amo! 
\end{minipage}



% Página de Agradecimentos
\cleardoublepage
\thispagestyle{empty}

\begin{center}

\textbf{AGRADECIMENTOS}
    
\end{center}

 Agradeço a todos da minha família pela paciência e compreensão. Nesse sentido, destaco minha mãe, Nilma, pelo incentivo e estímulo para enfrentar as barreiras da vida; meu irmão, Rodrigo, pelo apoio de sempre; minha filha, Amanda Grazyelle, por ser carinhosa e dedicada, tenho orgulho de ser sua mãe, pois com você aprendi o que é o amor verdadeiro; e meu marido, Paulo Estevão, pelo companheirismo, carinho e amor.

Agradeço aos colegas de trabalho do Sergipe Parque Tecnológico (SergipeTec), especialmente à diretoria, a José Augusto, Anízio, Luciana, Carlos e ao coordenador Rennan, pelo incentivo à realização do meu mestrado, pelas condições oferecidas e por permitirem minha dedicação à linha de pesquisa intitulada como \textit{Análise de Desempenho e Eficiência Energética de Usinas Fotovoltaicas}, agradeço também aos estagiários Ayslan, Gabriel e Gustavo, pela dedicação, comprometimento e colaboração, bem como pela troca de conhecimentos ao longo do desenvolvimento desta pesquisa.

Agradeço à Universidade Federal de Sergipe e ao Programa de Pós-Graduação em Engenharia Elétrica, por terem me proporcionado essa experiência durante a minha caminhada. Ao meu orientador, Prof. Dr. Douglas Bressan Riffel, agradeço pelos conselhos, ensinamentos, disponibilidade e valiosas contribuições que foram fundamentais para a realização desta pesquisa; Agradeço também aos professores que desempenharam com dedicação as aulas ministradas e pela contribuição para a minha formação.

Agradeço a todos que contribuíram direta e indiretamente para a minha trajetória, até as críticas foram essenciais para me incentivar a não desistir dos meus sonhos. O caminho seguido foi longo repleto de dificuldades para chegar até aqui. 

Por fim, agradeço a Deus por me guiar e colocar pessoas tão especiais ao meu lado.



% Página de Epígrafe
\cleardoublepage
\thispagestyle{empty}

\begin{center}
\textbf{EPÍGRAFE}
\end{center}

\vspace*{\fill}

\noindent
\hspace*{8cm}
\begin{minipage}{\dimexpr\textwidth-8cm\relax}
\textit{“Dizem que a vida é para quem sabe viver,
mas ninguém nasce pronto. \\
A vida é para quem é corajoso o suficiente para se arriscar
e humilde o bastante para aprender.”}

\begin{flushright}
- Clarice Lispector
\end{flushright}
\end{minipage}





% Página de Resumo
\cleardoublepage
\thispagestyle{empty}

\begin{center}
\textbf{RESUMO}
\end{center}


Diante de um cenário de desenvolvimento exponencial do setor de energias renováveis no Brasil, o ramo da energia solar fotovoltaica vem se destacando como uma competitiva fonte de geração de energia elétrica, assim crescendo a importância do monitoramento dos dados gerados pelas usinas para avaliar o desempenho, melhorar a eficiência energética e identificar possíveis falhas no sistema. 

O presente trabalho apresenta uma análise de performance baseada em coleta de dados de geração durante um intervalo de um ano, obtidos entre janeiro e dezembro de 2025, considerando o levantamento de dados nos 53 microinversores da usina solar fotovoltaica em estrutura do tipo carport instalada no Sergipe Parque Tecnológico (SergipeTec), localizado na cidade de São Cristóvão - Sergipe. A análise foi realizada por meio de estudo teórico e posteriormente com a criação de planilhas e cálculos de sete indicadores chave de desempenho (KPIs), onde referem-se a taxa de desempenho, o fator de utilização de capacidade, a produtividade final, o rendimento do arranjo, as perdas de captura, as perdas do sistema e a eficiência global do sistema. Estes KPIs foram calculados utilizando ferramenta computacional, tais como o Microsoft Excel.  O levantamento dos dados foram adquiridos na plataforma de monitoramento SOLARMAN Smart, enquato a criação de tabelas e gráficos foram executadas utilizando em conjunto o Microsoft Excel com o software Power BI. 

Ao longo do periodo analisado, observou-se que, dentre os 53 microinversores da usina, apenas uma faixa entre 3 e 8 microinversores operam com desempenho e produtividade abaixo dos demais, sendo por fatores externos como sombreamento, chuvas, sujeiras nas placas, mas o principal fator tem sendo o mal funcionamento de um ou mais módulos fotovoltaicos conectados ao microinversor analisado. Constatou-se que, nos meses com um volume maior de chuva em Sergipe, principalmente em maio, junho e julho, a incidência de radiação solar é muito menor e consequentemente o desempenho e a produtividade da usina decai em comparação com os outros meses em que a irradiação solar diária média é maior.

\noindent\textbf{Palavras Chave:} Energia solar fotovoltaica; Carport solar; Eficiência energética; Análise de performace; Indicadores chave de desempenho.



% Página de Abstract
\cleardoublepage
\thispagestyle{empty}

\begin{center}
\textbf{ABSTRACT}
\end{center}

Given the exponential growth of the renewable energy sector in Brazil, the photovoltaic solar energy sector has been standing out as a competitive source of electricity generation, thus increasing the importance of monitoring the data generated by the plants to evaluate performance, improve energy efficiency, and identify possible system failures.

This paper presents a performance analysis based on generation data collected over a one-year period, obtained between January and December 2025, considering data collected from the 53 microinverters of the carport-type photovoltaic solar power plant installed at the Sergipe Technological Park (SergipeTec), located in the city of São Cristóvão - Sergipe. The analysis was carried out through a theoretical study and subsequently with the creation of spreadsheets and calculations of seven key performance indicators (KPIs), which refer to the performance rate, capacity utilization factor, final productivity, array yield, capture losses, system losses, and overall system efficiency. These KPIs were calculated using computational tools such as Microsoft Excel. The data collection was acquired on the SOLARMAN Smart monitoring platform, while the creation of tables and graphs was performed using Microsoft Excel in conjunction with the Power BI software.

Throughout the analyzed period, it was observed that, among the 53 microinverters of the plant, only a range between 3 and 8 microinverters operate with performance and productivity below the others, due to external factors such as shading,
rain, dirt on the panels, but the main factor has been the malfunction of one or more photovoltaic modules connected to the analyzed microinverter. It was found that, in the months with a higher volume of rain in Sergipe, mainly in May, June and July, the incidence of solar radiation is much lower and consequently the performance and productivity of the plant decreases compared to the other months in which the average daily solar irradiance is higher.

\noindent\textbf{Keywords:} Solar photovoltaic energy; Solar carport; Energy efficiency; Performance analysis; Key performance indicators;